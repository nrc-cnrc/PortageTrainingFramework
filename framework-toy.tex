\documentclass[11pt]{article}
\usepackage{isolatin1}
\usepackage{xspace}

\newcommand{\phs}{\tild{s}}   % source phrase
\newcommand{\pht}{\tild{t}}   % target phrase
% Official typesetting of PORTAGEshared
\newcommand{\PS}{PORTAGE\emph{shared}\xspace}

\title{A toy experiment using the \PS \\
       experimental framework}
\date{}
\author{Eric Joanis}

\begin{document}
\maketitle

\begin{center}
An adaption of George Foster's \emph{Running Portage: A Toy Example} \\
to Samuel Larkin's experimental framework, updated to reflect recommended usage
of \PS.
\end{center}

\begin{center}
{~} \\ \tiny
   Technologies langagi{\`e}res interactives /
      Interactive Language Technologies \\
   Institut de technologie de l'information /
      Institute for Information Technology \\
   Conseil national de recherches Canada /
      National Research Council Canada \\
   Copyright \copyright 2008, Sa Majest{\'e} la Reine du Chef du Canada /
      Her Majesty in Right of Canada
\end{center}

\section{Introduction}

This document describes how to run a toy experiment using this experimental
framework from beginning to end.  It is intended as a tutorial in using \PS, as
well as a starting point for further experiments.  Although the framework
automates most of the steps described below, we go through them one by one here
in order to better explain how to use the \PS software suite.

\PS can be viewed as a set of programs for turning a bilingual corpus into a
translation system.  This document describes this process performed on a
trivially small data set.  The example is for French to English translation,
using text from the Hansard corpus.  It is too small for good translation, but
large enough to give the flavour of a more realistic setup. Running time is
about XXX.

\subsection{Running the toy experiment}

To begin, you must build or obtain the \PS executables and ensure that they are
in your path, typically by sourcing the \texttt{SETUP.bash/tcsh} file as
customized for your environment during installation of \PS.  You should also
set your environment variable \texttt{\$PORTAGE} to the directory where \PS is
installed, which is also done by \texttt{SETUP.bash/tcsh}.  Then you must make
a complete copy of this framework directory hierarchy, because it is designed
to work in place, creating the models within this hierarchy itself.

For example:
\begin{verbatim}
   > cd \$PORTAGE
   > cp -pr framework toy.experiment
   > cd toy.experiment
\end{verbatim}
All commands provided in the rest of this document are assumed to run in
toy.experiment or in a subdirectory thereof, which will all be specified
relative to toy.experiment.

As you work through the example, the commands that you are supposed to type are
preceded by a prompt, \texttt{>}, and the system's response is not, though
system output is not always fully reproduced here.  When it is, results
(especially numbers) may vary a little from the ones shown, due to platform
differences.

Now you need to drop a sentence aligned parallel corpus into the framework,
with training, dev and test data.  Ideally you need two dev sets, one for
tuning decoder weights and one for tuning rescoring weights.  You might skip
rescoring if speed is a concern, and you could use the same dev set in both
steps, but for this example we'll use two dev sets.  We often use two or more
test sets, but this is also optional.

We intentionally did not automate the steps required to create your train, dev
and test sets, because there is no generic formula that makes sense in all, or
even most cases.  If your data is chronological, such as news wires, you might
want to choose dev sets that are dated later than all training data, and test
sets dated later than all training and dev data.  This is what is most often
done, but it must done manually.  You could also take a random sample from your
training data for your dev data, but you run the risk of having dev data that
is more similar to your training data than to your test data, which might hurt
translation quality.  Ideally, your dev set should be representative of the
kind of text you will translate later, such as your test set.

Having said all this, we need these input files to run the example.  



\end{document}
