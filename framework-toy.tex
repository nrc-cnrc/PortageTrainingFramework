\documentclass[11pt]{article}
\usepackage{isolatin1}
\usepackage{xspace}

\newcommand{\phs}{\tild{s}}   % source phrase
\newcommand{\pht}{\tild{t}}   % target phrase
% Official typesetting of PORTAGEshared
\newcommand{\PS}{PORTAGE\emph{shared}\xspace}

\title{A toy experiment using the \PS \\
       experimental framework}
\date{}
\author{Eric Joanis}

\begin{document}
\maketitle

\begin{center}
An adaption of George Foster's \emph{Running Portage: A Toy Example} \\
to Samuel Larkin's experimental framework, updated to reflect recommended usage
of \PS.
\end{center}

\begin{center}
{~} \\ \tiny
   Technologies langagi{\`e}res interactives /
      Interactive Language Technologies \\
   Institut de technologie de l'information /
      Institute for Information Technology \\
   Conseil national de recherches Canada /
      National Research Council Canada \\
   Copyright \copyright 2008, Sa Majest{\'e} la Reine du Chef du Canada /
      Her Majesty in Right of Canada
\end{center}

\section{Introduction}

This document describes how to run a toy experiment using this experimental
framework from beginning to end.  It is intended as a tutorial in using \PS, as
well as a starting point for further experiments.  Although the framework
automates most of the steps described below, we go through them one by one here
in order to better explain how to use the \PS software suite.

\PS can be viewed as a set of programs for turning a bilingual corpus into a
translation system.  This document describes this process performed on a
trivially small data set.  The example is for French to English translation,
using text from the Hansard corpus.  It is too small for good translation, but
large enough to give the flavour of a more realistic setup. Running time is
about XXX.

\subsection{Running the toy experiment}

To begin, you must build or obtain the \PS executables and ensure that they are
in your path, typically by sourcing the \texttt{SETUP.bash/tcsh} file as
customized for your environment during installation of \PS.  You should also
set your environment variable \texttt{\$PORTAGE} to the directory where \PS is
installed, which is also done by \texttt{SETUP.bash/tcsh}.  Then you must make
a complete copy of this framework directory hierarchy, because it is designed
to work in place, creating the models within this hierarchy itself.  The
philosophy of this framework is that each experiment is done in a separate copy
of the framework.

For example:
\begin{verbatim}
   > cd \$PORTAGE
   > cp -pr framework toy.experiment
   > cd toy.experiment
\end{verbatim}
All commands provided in the rest of this document are assumed to run in
toy.experiment or in a subdirectory thereof, which will all be specified
relative to toy.experiment.

As you work through the example, the commands that you are supposed to type are
preceded by a prompt, \texttt{>}, and the system's response is not, though
system output is not always fully reproduced here, for brevity's sake.  When it
is, results (especially numbers) may vary a little from the ones shown, due to
platform differences.

Many of the commands are expressed as {\tt make} targets. This has the
advantage of requiring less typing, while still allowing you to see the actual
commands executed by the system, which are always echoed by {\tt make} (and
which of course you could also type directly). {\tt make} also lets you skip
steps (except for the first one, since it is done manually).  For example, if
you are not interested in any steps before decoder weight optimization, you can
go to section~\ref{COW} and type {\tt make tune} to begin at that point
(although you will have to wait until the system executes the necessary
commands to catch up). Here are some other useful make commands:
\begin{itemize}
\item {\tt make all} runs all remaining steps at any point.
\item {\tt make clean} cleans up and returns the directory to its initial state
\item {\tt make -j} {\em anything} builds the target {\em anything} by running
      commands in parallel whenever possible. This can be much faster if you
      have a multiprocessor machine.
\item {\tt make help} displays some help and the main targets available in the
      makefile.
\end{itemize}

\subsection{Overview of the Process}

Here is an overview of the \PS process, as described in the following
sections (section numbers are in brackets):
\begin{enumerate}
\item Corpus preparation (\ref{CorpusPreparation}): includes tokenization
      (\ref{Tokenization}), alignment (\ref{Alignment}), lowercasing
      (\ref{Lowercasing}), and splitting (\ref{Splitting}).
\item Model training (\ref{Training}): includes language model (\ref{LM}),
      translation model (\ref{TM}), truecasing model (\ref{TC}), decoder weight
      optimization (\ref{COW}), and rescoring model training (\ref{RAT}).
\item Translating and testing (\ref{TranslatingTesting}): includes
      translating (\ref{Translating}), truecasing (\ref{Truecasing}) and
      testing (\ref{Testing}).
\end{enumerate}
These steps are fairly standard, but of course there are many variants on the
sample process illustrated here. For more information, see the Portage HTML
user documentation. For detailed information on a particular program, run it
with the {\tt -h} flag.

\section{Corpus Preparation} \label{CorpusPreparation}

Corpus preparation involves converting raw document pairs into tokenized,
sentence-aligned files in the format expected by Portage.  We provide a
tokenizer, a sentence aligner and sample corpus pre-processing scripts with
\PS, which can help you with these steps.  For details, see section \emph{Text
Processing} in the user manual (found in
\textt{PORTAGEshared/doc/user-manual.html} on the CD).

\subsection{Splitting the Corpus} \label{Splitting}

The tokenized, sentence-aligned corpus must be split into separate portions
in order to run experiments. Distinct, non-overlapping sub-corpora are required
for model training (see section~\ref{Training}), for optimizing decoder weights
(section~\ref{COW}), for training a rescoring model (section~\ref{RAT}), and
one  testing (section~\ref{Testing}). Typically, the latter three corpora are
around 1000 segments each.  If the corpus is chronological, then it is a good
idea to choose the test corpus from the most recent material, which is likely
to resemble future text more closely.

Since proper splitting of your corpus needs to take into account the structure
and nature of your data, we don't perform those steps in this toy experiment.
Instead, we've provided small sets that can be found here:
\texttt{PORTAGEshared/test-suite/unit-testing/framework-toy}.  These sets
are very small to reduce running time, so you should not be surprised when the
results of this toy experiment are of poor quality.

To drop these sets into the framework, simply copy them (or make symbolic
links) into your copy's corpora directory:
\begin{verbatim}
   > cp $PORTAGE/test-suite/unit-testing/framework-toy/*.al corpora/
   > wc corpora/*.al
       100    2115   11589 corpora/dev1_en.al
       100    2330   13680 corpora/dev1_fr.al
       100    2095   11462 corpora/dev2_en.al
       100    2443   13905 corpora/dev2_fr.al
       100    2379   13292 corpora/test_en.al
       100    2616   15622 corpora/test_fr.al
      8992  184569  994516 corpora/train_en.al
      8992  211240 1194385 corpora/train_fr.al
     18584  409787 2268451 total
\end{verbatim}

\subsection{Setting framework parameters} \label{FrameworkParams}

Now you need to tell the framework what you've called your test set, since we
didn't use the default.  Edit \texttt{Makefile.params} and change the
\texttt{TRANSLATE\_SET} variable to say
\texttt{test} instead of \texttt{test1 test2}.
This variable lists the stem of all your test sets, and the framework
assumes you have two by default.
You will also need to set \texttt{TRAIN\_LM} to \texttt{train} instead of
\texttt{lm-train} since we won't be using a separate corpus to train language
models with.
After editing the file:
\begin{verbatim}
   > grep "export TRANSLATE_SET" Makefile.params
   export TRANSLATE_SET ?= test
   > grep "export TRAIN_LM" Makefile.params
   export TRAIN_LM      ?= train
\end{verbatim}

While you're there, you should get familiar with all the variables in the
\emph{User definable variables} section of this file.  It is where most of the
configurable behaviours are set, such as whether to do rescoring and/or
truecasing, what LM toolkit you are using (if any), etc.  

In this toy experiment, we'll use the default value for most parameters, but
you need to set the \texttt{IRSTLM} variable if you have installed IRSTLM and
wish to use it, or the \texttt{LM\_TOOLKIT\_SRI} to use SRILM (if your
licensing requirements permit it).  If you have neither, leave both variables
undefined; we provide trained language models that you can drop into the
framework to finish this toy experiment while skipping language model training.

\subsection{Lowercasing and adding escapes}

To reduce data sparseness, we convert all files to lowercase.
\begin{verbatim}
   > cd corpora
   > make lc
   cat dev1_en.al | lc-utf8.pl > dev1_en.lc
   [...]
\end{verbatim}

The decoder, canoe, treats \texttt{<}, \testtt{>} and \texttt{\\} as special
characters, so we need a copy of the dev and test files with these characters
escaped, in the source language only.

\begin{verbatim}
   > make rule
   canoe-escapes.pl -add dev1_en.lc > dev1_en.rule
   canoe-escapes.pl -add dev2_en.lc > dev2_en.rule
   canoe-escapes.pl -add test_en.lc > test_en.rule
\end{verbatim}



\section{Training} \label{Training}

This step creates various models and parameter files that are required for
translation. There are five steps in training: creating a language model,
creating a truecasing model, creating translation models, optimizing decoder
weights, and creating a rescoring model.

\subsection{Creating a Language Model} \label{LM}

Portage does not come with programs for language model training. However,
it accepts models in the widely-used ``DARPA'' format which is produced by popular
toolkits such as CMU SLM (www.speech.cs.cmu.edu/ SLM\_info.html), SRILM
(www.speech.sri.com/projects/srilm), or IRSTLM (irstlm.sourceforge.net). Here
is the SRILM command used to produce the model {\tt train\_en.lm} supplied with
this example:
\begin{verbatim}
   > make lm
   ngram-count -interpolate -kndiscount1 -kndiscount2 -kndiscount3 \
      -order 3 -text train_en.al -lm train_en.lm
\end{verbatim}
This assumes that the target language is English; for English to French
translation, a similar command could be used with {\tt train\_fr.al} to create
a French language model.


\end{document}
